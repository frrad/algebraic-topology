\documentclass[11pt]{article}



\usepackage[all]{xy}
\usepackage{fancyhdr}
\usepackage{amsthm}
\usepackage{amssymb}
\usepackage{setspace}
\pagestyle{fancyplain}

\begin{document}

\lhead{Frederick Robinson}
\rhead{Independent Study}



\title{Hatcher}
\author{Frederick Robinson}
\date{14 February 2010}
\maketitle


\section*{Hatcher 4.1}
\section{Question 16}
\subsection{Question}
Show that a map $f : X \to Y$ between connected CW complexes factors as a composition $X \to Z_n \to Y$ where the first map induces isomorphisms on $\pi_i$ for $i \leq n$ and the second map induces isomorphisms on $\pi_i$ for $i \geq n +1$.
\subsection{Answer}

Adapt Example 4.17 on page 354. 

Construct $Z_n$ as follows. Start with a copy of $X$, then add $(n+1)$-cells to it corresponding to generators of $\pi_{n+1}(Y)$.  This space has a $\pi_{n+1}$ that is too large but by adding $n+2$-cells we can get the appropriate group.

 These $n+2$-cells (as well as the just-added $n+1$ cells) map to $Y$. by taking generators of $\pi_{n+1}$ to generators in $Y$, and relations to relations.   Repeating this process infinitely many times, we are done.
\section*{Hatcher 4.2}

\section{Question 1}
\subsection{Question}Use homotopy groups to show there is no retraction $\mathbb{R}P^n \to \mathbb{R}P^k$ if $n>k>0$.
\subsection{Answer}
First we'll make some calculations of the homotopy groups of $\mathbb{R}P^n$. 

\begin{proof}Recall that we have the fibration
\[S^0 \to  S^n \to \mathbb{R}P^n\]
for all $n$. So, we may write the long exact sequence of the fibration as
\[\cdots \to \pi_k(S^0) \to \pi_k(S^n) \to \pi_k(\mathbb{R}P^n) \to \pi_{k-1}(S^0) \to \cdots \to \pi_0(S^n) \to 0.\]
Clearly, 
\[\pi_n(S^0) = \left\{ \begin{array}{ll} 0 & n \neq 0 \\ \mathbb{Z}/2\mathbb{Z} & n =0 \end{array}\right. . \]
Thus, $k> 1$, $\pi_k(\mathbb{R}P^n)$ fits into the exact sequence
\[0 \to \pi_k(S^n) \to \pi_k(\mathbb{R}P^n) \to 0\]
and we have $\pi_k(S^n) \cong \pi_k(\mathbb{R}P^n)$ for $k>1$. At the end of this long exact sequence we have
\[\cdots \to \pi_1(S^n) \to \pi_1(\mathbb{R}P^n) \to \pi_0(S^0) \to \pi_0(S^n) \to 0\]
\[\cdots \to \pi_1(S^n) \to \pi_1(\mathbb{R}P^n) \to \mathbb{Z}/2\mathbb{Z} \to \pi_0(S^n) \to 0.\]
Assuming that $n > 1$, this is just
\[ 0 \to \pi_1(\mathbb{R}P^n) \to \mathbb{Z}/2\mathbb{Z} \to 0\]
and $\pi_1(\mathbb{R}P^n ) \cong \mathbb{Z} / 2 \mathbb{Z}$ for $n>1$. Since $\mathbb{R}P^1 \cong S^1$ we have $\pi_1(\mathbb{R}P^1) \cong \pi_1(S^1) \cong \mathbb{Z}$ in the case of $n=1$.

So, by substituting in the values of $\pi_k(S^n)$ where appropriate we have a table of homotopy groups that looks like
\[\begin{array}{|c|c|c|c|c|c|c|c}
\hline
&\pi_1&\pi_2&\pi_3&\pi_4&\pi_5&\pi_6& \cdots\\
\hline
\mathbb{R}P^1 &\mathbb{Z}&0&0&0&0&0 &\cdots \\
\hline
\mathbb{R}P^2 & \mathbb{Z}/2\mathbb{Z} &\mathbb{Z}&\mathbb{Z}&\mathbb{Z}/2\mathbb{Z}&\mathbb{Z}/2\mathbb{Z}&\mathbb{Z} /12\mathbb{Z}&\cdots\\
\hline
\mathbb{R}P^3 &  \mathbb{Z}/2\mathbb{Z}&0&\mathbb{Z}&\mathbb{Z}/2\mathbb{Z}&\mathbb{Z}/2\mathbb{Z}& \mathbb{Z}/12\mathbb{Z}&\cdots\\
\hline
\mathbb{R}P^4 &  \mathbb{Z}/2\mathbb{Z}&0&0&\mathbb{Z}&\mathbb{Z}/2\mathbb{Z}&\mathbb{Z}/2\mathbb{Z} &\cdots\\
\hline
\mathbb{R}P^5 &  \mathbb{Z}/2\mathbb{Z}&0&0&0&\mathbb{Z}&\mathbb{Z}/2\mathbb{Z} &\cdots\\
\hline
\mathbb{R}P^6 & \mathbb{Z}/2\mathbb{Z} &0&0&0&0&\mathbb{Z} &\cdots \\
\hline
\vdots&\vdots&\vdots&\vdots&\vdots&\vdots&\vdots&\ddots\\
\end{array}\]
\end{proof}

Now, it is easy to see that there is no  retraction $\mathbb{R}P^n \to \mathbb{R}P^k$ if $n>k>0$.
\begin{proof}
Suppose towards a contradiction that there is such a retraction $r: \mathbb{R}P^n \to \mathbb{R}P^k $.  But then we should be able to factor the group $\pi_k(\mathbb{R}P^k)$ by
\[\pi_k(\mathbb{R}P^k) \stackrel{i_*}{\to} \pi_k(\mathbb{R}P^n) \stackrel{r_*}{\to} \pi_k (\mathbb{R}P^k)\]
first including, then retracting. This is a contradiction, since the middle group is either 0, or $\mathbb{Z}/2\mathbb{Z}$ while the outer one is $\mathbb{Z}$.
\end{proof}

\section{Question 2}
\subsection{Question}
Show the action of $\pi_1(\mathbb{R}P^n)$ on $\pi_n(\mathbb{R}P^n) \cong \mathbb{Z}$ is trivial for $n$ odd and nontrivial for $n$ even.
\subsection{Answer}
Let's consider $S^n \subset \mathbb{R}^{n+1}$ the unit sphere.  Then  $\pi_n( S^n)$ is generated by the identity map on the sphere, and $-1 \in \pi_n( S^n)$ is generated by any negative determinant member of $SO(n+1)$ (which fixes the basepoint). Furthermore, $\mathbb{R}P^n$ is obtained by associating antipodal points in the sphere. That is, $x  \sim y$ if $x = -I y$ for $I$ the identity matrix in $SO(n+1)$. Denote by $p$ the map which sends $S^n \to \mathbb{R}P^n$ by this association. 

Since $\pi_1(\mathbb{R}P^n)=\mathbb{Z}/2$ it suffices to check that the nontrivial element of $\pi_1(\mathbb{R}P^n$ sends the generator of $\pi_n(\mathbb{R}P^n)$ to itself with $n$ odd and to $-1$ for $n$ even. This nontrivial element has a representative whose preimage under $p$ is a great circle in $S^n$ going through the base point of our homotopy.



Assuming without loss of generality that this basepoint has coordinate $(1,0,0,\dots)$, we see that acting by this element induces a (not basepoint preserving) rotation in our sphere, exchanging the basepoint, with the opposite pole. That is, it acts by the matrix
\[A = \left( \begin{array}{ccccc}-1\\ & 1 \\  &  & \ddots \\ &&&1\\ &&&&-1 \end{array} \right).\]
(Note: there is nothing special about this particular matrix, it just needs to have -1 in the first entry, and determinant 1.)

Now, taking into account that anitpodal points are associated, we see the desired result. A pair, $(x, -Ix)$ is taken by this map to $A(x,-Ix) = (Ax, -Ax)=-A(-Ix, x)$. 
In the case that $n+1$ is odd, $-A$ has negative determinant, however if $n+1$ is even, $-A$ has positive determinant. Since 

\section{Question 6}
\subsection{Question}
Show that the relative form of the Hurewicz Theorem in dimension $n$ implies the absolute form in dimension $n-1$ by considering the pair $(CX,X)$ where $CX$ is the cone on $X$.
\subsection{Answer}
The Hurewicz Theorem states:

If $(X, A)$ is an $(n-1)$-connected pair of path-connected spaces with $n\geq2$ and $A\neq 0$ then $h':\pi_n'(X,A,x_0) \to H_n(X,A)$ is an isomorphism and $H_i(X,A) =0$ for $i<n$.

The absolute version is just the result of taking $A$ to be the basepoint. Namely:

If $X$ is an $(n-1)$-connected space with $n\geq2$ then $h':\pi_n'(X,x_0) \to H_n(X)$ is an isomorphism and $H_i(X) =0$ for $i<n$.

\begin{proof}
Let $X \neq 0$ be an $(n-2)$-connected space for $n \geq 2$. Then, as the pair $(CX, X)$ is $(n-1)$-connected, and each is path connected, $X\neq 0$, $h':\pi'_n(CX,X,x_0) \to H_n(CX,X)$ is an isomorphism and $H_i(CX,X)=0$ for $i<n$ by the relative Hurewicz Theorem.

However this implies that  we have $H_i(X)=0$ for $i< n-1$ by the long exact sequence of relative homology groups. the fact that $h':\pi'_n(CX,X,x_0) \to H_n(CX,X)$ is an isomorphism implies that $h':\pi'_n(X,x_0) \to H_n(X)$ is an isomorphism in one lower dimension, since $\pi_n(CX,X) \cong \pi_{n-1}(X)$, and similarly for homology groups.
\end{proof}

\section{Question 7}
\subsection{Question}
Construct a $CW$ complex $X$ with prescribed homotopy groups $\pi_i(X)$ and prescribed actions of $\pi_1(X)$ on the $\pi_i(X)$'s.
\subsection{Answer}


\end{document}
